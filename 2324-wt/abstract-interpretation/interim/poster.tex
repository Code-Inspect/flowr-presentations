\makeatletter
\newcommand\input@path{{smile},{flowr-logo}}
\makeatother

\documentclass[color,coloraccent=red!60!black]{poster}

\usepackage[notext,color=accent]{flowrlogo}
\usepackage{lipsum}
\usepackage[normalem]{ulem}

\title{Abstract Interpretation}
\subtitle{The cooler AI}
\author{Lukas Pietzschmann, Florian Sihler}
\uni{Ulm University}
\institute{Institute of Software Engineering and Programming Languages}
\date{02/02/2024}
\logo{logos/splogo.png}

\columnseprule0pt
\def\ULthickness{.7pt}

\def\flowr{\textit{flowR}}

\begin{document}
\tikzset{logo/.append style={fill opacity=0.85}}
\begin{tikzpicture}[remember picture, overlay]
	\node[shift={(-18mm,-18mm)},node on layer=background,scale=13] at (current page.north east) {\flowrlogo};
\end{tikzpicture}
\maketitle

\begin{tikzpicture}
	\node[roundednode,minimum width=\textwidth,minimum height=0.25\textheight,draw=black,fill=lightgray!20,fill opacity=0.85] {\Large Eye catching figure};
\end{tikzpicture}

\begin{multicols}{3}
	\begin{minipage}{\dimexpr2\columnwidth+\columnsep\relax}
		\section*{What even is \flowr}
		\lipsum[4]
		\section*{Abstract Interpretation}
		\begin{tikzpicture}
			\node[roundednode,minimum width=\textwidth,minimum height=0.2\textheight,draw=black,fill=lightgray!20] {\Large Maybe an example?!};
		\end{tikzpicture}
		\lipsum[1-3]
	\end{minipage}\vfill\columnbreak\null\columnbreak
	\section*{Project Organization}
	\paragraph{Organization} We're doing \emph{weekly meetings}, where we discuss the
	progress made in the past week, talk about open issues, and prioritize tasks for the
	next week. We also keep a \emph{record of important things} that came up in our
	meeting. If there's any spare time, Florian often tells me about new
	\sout{weird}cool thing he learned about the R language.
	\paragraph{Tech Stack} \flowr{} is developed with \emph{TypeScript}, then compiled
	down to JavaScript with Node.js as its runtime. While we use different libraries for
	utilitarian tasks --- like chai for assertions, mocha for tests, or tslog for
	logging --- all major functionalities are implemented \emph{by hand}. This includes
	the abstract interpretation itself.
	\paragraph{QA} To ensure that we don't break stuff, \flowr{} features over
	\emph{1000 tests}. We're also constantly adding new ones, as soon as there is new
	code in the project.\par To ensure the employment of best practices, we always do a
	\emph{code review} on pull requests. This massively helps with keeping the code easy
	to understand and well readable.\par
	We also make heavy use of \emph{assertions} --- or how we call them:~guards ---
	whenever possible to make sure our mental model aligns with the actual execution of
	the code.
	\paragraph{Documentation} \flowr's documentation is split into two parts:
	\begin{enumerate*}
		\item a \emph{user facing} documentation hosted in a GitHub wiki and
		\item a \emph{developer facing} documentation built from inline comments
	\end{enumerate*}. % TODO more on docs
	\section*{Future Work}
	\lipsum[2]
\end{multicols}
\end{document}