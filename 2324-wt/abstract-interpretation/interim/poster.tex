\makeatletter
\newcommand\input@path{{smile},{flowr-logo}}


\documentclass[color,coloraccent=red!60!black,listings]{poster}

\usepackage[notext,color=accent]{flowrlogo}
\usepackage{lipsum}
\usepackage[normalem]{ulem}
\usepackage[verbatim]{lstfiracode}
\usepackage[notext,nott,nosf,nomath]{kpfonts-otf}
\usepackage{stmaryrd}

\title{Abstract Interpretation}
\subtitle{The cooler AI}
\author{Lukas Pietzschmann, Florian Sihler}
\uni{Ulm University}
\institute{Institute of Software Engineering and Programming Languages}
\date{02/02/2024}
\logo{logos/splogo.png}

% \overfullrule=1mm

\columnseprule0pt
\def\ULthickness{.7pt}

\colorlet{smile@lst@color@lnnr}{darkgray}

\def\flowr{\textit{flowR}}

\makeatletter
\lstdefinestyle{firastyleb}{style=FiraCodeStyle,style=smile@lst@base}
\lstdefinestyle{firastylep}{style=FiraCodeStyle,style=smile@lst@plain}

\lstset{tabsize=4,style=firastyleb}

\def\rc#1{\lstinline[language=r]{#1}}
\setmathfont[version=kp]{KpMath}
\def\tbot{{\mathversion{kp}$\bot$}}
\def\ttop{{\mathversion{kp}$\top$}}

\newsavebox\flowrlogob
\savebox\flowrlogob{\flowrlogo}
\def\uselogo#1{\scalebox{#1}{\usebox\flowrlogob}}
\begin{document}
\tikzset{logo/.append style={fill opacity=0.9}}
\begin{tikzpicture}[remember picture, overlay]
	\node[shift={(-18mm,-18mm)},node on layer=background] at (current page.north east) {\uselogo{13}};
\end{tikzpicture}
\maketitle
\makeatletter\tikzset{
	lw/.style={line width=.5\smile@linewidth},
	n/.style={minimum width=7mm,minimum height=7mm,shadow},
	a/.style={arrow,short=1mm},
	no/.style={opacity=0.3,no shadows},
	t/.style={roundednode,dashed,node on layer=background,inner sep=2mm,fill=lightgray!93!blue!20,fill opacity=0.9,minimum width=\columnwidth-\smile@linewidth,minimum height=4.2cm},
	dr/.style={pill},
	ds/.style={pill,shift={(1mm,1mm)}},
	label/.style={line join=round,chamfered rectangle,chamfered rectangle xsep=2pt,lw,inner sep=2pt,draw=black,fill=white},
}

\begingroup
\makeatletter
\tikzset{node distance=6mm}
\begin{multicols}{3}
	\def\i#1{\color{gray!50!black}\textsubscript{#1}}
	\def\j#1{\rlap{\i{#1}}}
	\newsavebox\exampleb
	\begin{lrbox}\exampleb
	\begin{lstlisting}[language=R,style=firastylep]
x§\j1§ <- 10
y§\j1§ <- 5 + x§\i2§
if(y§\j2§ > 10) {
	compA(y§\i3§)
} else {
	compB(y§\i4§)
}
	\end{lstlisting}
	\end{lrbox}
	\begin{tikzpicture}
		\node (L) {\usebox\exampleb};
		\node[t,fit=(L)] (F) {};
		% \node[label] at (F.south) {Step 1};
	\end{tikzpicture}\par
	\paragraph{Step 1} To set everything up, we retrieve the code, get its abstract
	syntax tree, normalize it, and finally build and return the dataflow graph. This
	graph will then be used as a base for the next step.\par
	\newcolumn
	\begin{tikzpicture}
		\node[n,roundednode] (X1) {\rc{x}\textsubscript1};
		\node[n,roundnode,right=of X1] (X2) {\rc{x}\textsubscript2};
		\node[n,roundednode,right=of X2] (Y1) {\rc{y}\textsubscript1};
		\node[n,roundnode,right=of Y1] (Y2) {\rc{y}\textsubscript3};
		\node[n,squarenode,right=of Y2] (C1) {\rc{compA}};

		\node[n,roundnode,below=of Y2] (Y3) {\rc{y}\textsubscript4};
		\node[n,squarenode,below=of C1] (C2) {\rc{compB}};

		\node[n,roundnode,above=of Y1] (Y4) {\rc{y}\textsubscript2};

		\draw[a] (X1) to (X2);
		\draw[a] (X2) to (Y1);
		\draw[a] (Y1) to (Y2);
		\draw[a] (Y2) to (C1);

		\draw[a] (Y1) to (Y3);
		\draw[a] (Y3) to (C2);

		\draw[a] (Y1) to (Y4);

		\node[dr] at (X1.north east) {\tiny$[10, 10]$};
		\node[ds] at (X2.north east) {\tiny$[10, 10]$};
		\node[dr] at (Y1.north east) {\tiny$[15, 15]$};
		\node[ds] at (Y2.north east) {\tiny$[15, 15]$};
		\node[ds] at (Y3.north east) {\tiny\tbot};
		\node[ds] at (Y4.north east) {\tiny$[15, 15]$};

		\node[t,fit=(X1)(X2)(Y1)(Y2)(Y3)(Y4)(C1)(C2)] (F) {};
		% \node[label] at (F.south) {Step 2};
	\end{tikzpicture}\par
	\paragraph{Step 2} In the second step, the actual abstract interpretation happens.
	This step gradually decorates the dataflow graph with the domains of the respective
	variables, by traversing the control flow graph.\par
	\columnbreak
	\begin{tikzpicture}
		\node[n,roundednode] (X1) {\rc{x}\textsubscript1};
		\node[n,roundnode,right=of X1] (X2) {\rc{x}\textsubscript2};
		\node[n,roundednode,right=of X2] (Y1) {\rc{y}\textsubscript1};
		\node[n,roundnode,right=of Y1] (Y2) {\rc{y}\textsubscript3};
		\node[n,squarenode,right=of Y2] (C1) {\rc{compA}};

		\begin{scope}[no,transparency group]
			\node[n,roundnode,below=of Y2] (Y3) {\rc{y}\textsubscript4};
			\node[n,squarenode,below=of C1] (C2) {\rc{compB}};
			\node[ds] at (Y3.north east) {\tiny\tbot};
		\end{scope}

		\node[n,roundnode,above=of Y1] (Y4) {\rc{y}\textsubscript2};

		\draw[a] (X1) to (X2);
		\draw[a] (X2) to (Y1);
		\draw[a] (Y1) to (Y2);
		\draw[a] (Y2) to (C1);

		\begin{scope}[no,transparency group]
			\draw[a] (Y1) to (Y3);
			\draw[a] (Y3) to (C2);
		\end{scope}

		\draw[a] (Y1) to (Y4);

		\node[dr] at (X1.north east) {\tiny$[10, 10]$};
		\node[ds] at (X2.north east) {\tiny$[10, 10]$};
		\node[dr] at (Y1.north east) {\tiny$[15, 15]$};
		\node[ds] at (Y2.north east) {\tiny$[15, 15]$};
		\node[ds] at (Y4.north east) {\tiny$[15, 15]$};

		\node[t,fit=(X1)(X2)(Y1)(Y2)(Y3)(Y4)(C1)(C2)] (F) {};
		% \node[label] at (F.south) {Step 3};
	\end{tikzpicture}\par
	% \paragraph{Step 3} The last step uses the decorated dataflow graph to determine wich
	% path can never be executed. The identified path will then be discarded from the
	% graph.\par
	\paragraph{Step 3} The last step uses the decorated dataflow graph to determine and
	track different properties of the original code. We can, e.g., catch endless loops,
	or, as shown above, ignore unreachable nodes in the dataflow graph.\par
\end{multicols}\bigskip
\endgroup

\begin{multicols}{3}
	\newlength\currentparskip\setlength\currentparskip\parskip
	\begin{minipage}{\dimexpr2\columnwidth+\columnsep\relax}
		\setlength\parskip\currentparskip%
		\section*{How Abstract Interpretation fits into \flowr}
		\paragraph{\flowr}\flowr{} is a static \emph{dataflow analyzer} and
		\emph{program slicer} for the R language. Program slicers reduce a given program
		to a set of statements --- the so called slices --- that influence a variable at
		a specific point in the program: the slicing criterion. \flowr{} formulates
		the slicing as a reachability problem on the dataflow graph, which itself is
		based on the abstract syntax tree of the given R~program.
		\paragraph{Abstract Interpretation} Abstract interpretation is all about
		figuring out \emph{runtime properties} of a given program without actually
		executing it. These properties include aspects like the sign or nullness of a
		variable. For now we decided to limit ourselves to determining the \emph{domain
		of numeric variables}. In other words, we want to know what values a variable
		can have at a certain point in the program.
		\paragraph{Abstract Interpretation in \flowr} Abstract interpretation helps
		\flowr{} to further \emph{narrow down the slice} of a program. This can be done
		by removing certain paths from the dataflow graph if we can prove that they will
		never be executed. Imagine a conditional like \rc{if(i < 10) doit() else run()}.
		If, through abstract interpretation, we know that \rc{i}'s value is always less than
		\rc{10}, there's no need to include the path stemming from the \rc{if}'s
		else-case in the slice. % TODO: is this okay? Or does it read like we only catch dead code
		\section*{Abstract Interpretation}
		\begingroup
		\lstset{numbers=left,numberblanklines=false,firstnumber=0,style=firastylep,numbers=left,numbersep=5pt}
		\let\origthelstnumber\thelstnumber
		\def\nolnnr{\lst@AddToHook{OnNewLine}{\let\thelstnumber\relax\advance\c@lstnumber-\@ne\relax}}
		\def\yeslnnr#1{%
			\setcounter{lstnumber}{\numexpr#1-1\relax}
			\lst@AddToHook{OnNewLine}{\let\thelstnumber\origthelstnumber\refstepcounter{lstnumber}}%
		}
		\def\lst@PlaceNumber{%
			\ifnum\value{lstnumber}=0\else%
			\llap{\normalfont\lst@numberstyle{\thelstnumber}\kern\lst@numbersep}\fi%
		}
		\newsavebox\codeexampleb
		\begin{lrbox}\codeexampleb
			\begin{lstlisting}[language=R]
§$\text{\normalfont let}\ a \in [0, 200]$\nolnnr§
§$\text{\normalfont let}\ b \in [5, 10]$\yeslnnr1\medskip§
if (a < 100) {
	a <- a + 1
} else {
	a <- a + 1
	b <- b + 2
}
a; b
			\end{lstlisting}
		\end{lrbox}
		\def\intv#1#2#3{$\left[#2, #3\right]$}
		\newsavebox\tableexampleb
		\begin{lrbox}\tableexampleb
			\def\arraystretch{1.1}\def\lnnr#1{\makeatletter{\smile@lst@style@linenr\llap{#1\hspace{2ex}}}}
			\makeatletter\begin{tabular}{ll}
				\noalign{\global\arrayrulewidth=.5\smile@linewidth}
				\hline
				\rowcolor{lightgray}  $a$ & $b$                             \\\hline
				\lnnr0\intv{a}{0}{200}    & \intv{b}{5}{10} \\
				\lnnr1\intv{a}{0}{99}     & \intv{b}{5}{10} \\
				\lnnr2\intv{a}{1}{100}    & \intv{b}{5}{10} \\
				\lnnr3\intv{a}{100}{200}  & \intv{b}{5}{10} \\
				\lnnr4\intv{a}{101}{201}  & \intv{b}{5}{10} \\
				\lnnr5\intv{a}{101}{201}  & \intv{b}{7}{12} \\
				\lnnr7\intv{a}{1}{201}    & \intv{b}{5}{12} \\\hline
			\end{tabular}
		\end{lrbox}
		\begin{multicols}{2}
			\begin{tikzpicture}
				\node[t,minimum height=5cm] (L) {\usebox\codeexampleb};
			\end{tikzpicture}\par
			\paragraph{Fig. 1} The code we want to analyze.\par\columnbreak
			\begin{tikzpicture}
				\node[t,minimum height=5cm] (L) {\usebox\tableexampleb};
			\end{tikzpicture}\par
			\paragraph{Fig. 2} The computed domain for both variables at every line.
		\end{multicols}\medskip
		We assume that the integer variables \rc{a} and \rc{b} have both been defined
		and initialized elsewhere. When entering the snippet shown above, their values
		are assumed to fall within the domain \intv{a}{0}{200} and
		\intv{b}{5}{10}.\par
		We then start by traversing the control flow graph in execution order,
		processing each element of the original program. We're especially interested in
		\begin{enumerate*}
			\item operations that change the value of a variable,
			\item assignments, and
			\item structures that influence the control flow
		\end{enumerate*}. When encountering an expression of type (1)~--- like addition,
		subtraction, and other arithmetic operations~--- we merge the operands' domains
		according to the given operation. We can observe this in line 2, where \rc{a}'s
		value was previously determined to be between $0$ and $99$. Adding \rc1 to
		\rc{a} modified its domain by incrementing both the lower and upper bound to $1$
		and $100$. When we encounter an assignment, we use the domain of the assignments
		source expression and map it to the target variable. We can also see this in
		line 2, where the domain of \rc{a + 1}~--- namely \intv{a}{1}{100}~--- is set as
		\rc{a}'s new domain. Lastly, if we encounter a structure of type (3), like
		conditionals or loops, we check how the condition narrows down a variable's
		domain. In the \rc{if}-\rc{else} above, we can be sure that in the then-case,
		the condition \rc{a < 100} will always hold, the opposite is true for the
		else-case. Figure~2 shows this in line 1, where \rc{a}'s domain \intv{a}{0}{200}
		got narrowed down to \intv{a}{0}{99}.\par Besides the domains shown above, we
		use two special values to indicate two cases. We use \ttop~(top) whenever the
		domain of a variable is unknown. Or in different words: when the variable could
		take every possible value. If the opposite is the case --- if a variable can't
		take any value --- we use \tbot~(bottom).
		\endgroup
	\end{minipage}\vfill\columnbreak\null\columnbreak
	\section*{Project Organization}
	\paragraph{Organization} We're doing \emph{weekly meetings}, where we discuss the
	progress made in the past week, talk about open issues, and prioritize tasks for the
	next week. We also keep a \emph{record of important things} that came up in our
	meeting. If there's any spare time, Florian often tells me about new
	\sout{weird}cool things he learned about the R language.
	\paragraph{Tech Stack} \flowr{} is developed with \emph{TypeScript}, then transpiled
	to JavaScript with Node.js as its runtime. While we use different libraries for
	utilitarian tasks --- like chai for assertions, mocha for tests, or tslog for
	logging --- all major functionalities, including abstract interpretation, are
	implemented \emph{by hand}, as there's a big lack of libraries offering these
	features for the R programming language.
	\paragraph{QA} To ensure the employment of best practices, we always do \emph{code
	reviews} on pull requests. This massively helps with keeping the code easy to
	understand and well readable.\par
	If an error slips through during the review, one of \flowr's extensive suite of over
	\emph{1000 tests} will probably catch it. Additionally, we consistently introduce
	new tests promptly whenever there is new code in the project.\par
	We also make heavy use of \emph{assertions} --- or how we call them:~guards ---
	whenever possible to make sure our mental model aligns with the actual execution of
	the code.
	\paragraph{Documentation} \flowr's documentation is split into two parts:
	\begin{enumerate*}
		\item a \emph{user facing} documentation hosted in a GitHub wiki and
		\item a \emph{developer facing} documentation built from inline comments
	\end{enumerate*}. % TODO more on docs
	\section*{Future Work}
	Well, there's still \emph{a lot to do}! As a first step, we need to implement
	support for \emph{more control structures}, \emph{operators}, and \emph{function
	calls}. When this is done and basic cases are handled well, we can start
	\emph{modifying the dataflow graph} and automatically through that, narrow down the
	slices \flowr{} produces. As a last step, there are many things that can be
	\emph{optimized}, like the representation of domains. While this is definitely not
	mission critical, good performance is always a nice thing to have.\par
	And all of this is only just the tip of the iceberg. Abstract interpretation allows
	for way more than just the tracking of value domain. Obvious extensions of this
	project could include the tracking of domains for different types, or pointer
	analysis.
\end{multicols}
\end{document}